
\documentclass[11pt,fleqn]{memoir} % Default font size and left-justified equations

\usepackage[top=3cm,bottom=3cm,left=3.2cm,right=3.2cm,headsep=10pt,a4paper]{geometry} % Page margins

\usepackage{xcolor} % Required for specifying colors by name
\definecolor{ocre}{RGB}{52,177,201} % Define the orange color used for highlighting throughout the book

\author{Laura Sakurai}
\title{The important of doctrine in praxis for the anarchist and communist revolution}
% Font Settings
%\usepackage{avant} % Use the Avantgarde font for headings
%\usepackage{mathptmx} 

%\usepackage{microtype} % Slightly tweak font spacing for aesthetics
\usepackage[utf8]{inputenc} % Required for including letters with accents
%\usepackage[T1]{fontenc} % Use 8-bit encoding that has 256 glyphs
\usepackage{epigraph}
% Bibliography
\usepackage[style=alphabetic,sorting=nyt,sortcites=true,autopunct=true,babel=hyphen,hyperref=true,abbreviate=false,backref=true,backend=biber]{biblatex}
\addbibresource{bibliography.bib} % BibTeX bibliography file
\defbibheading{bibempty}{}
\usepackage{enumitem}

%\input{structure} % Insert the commands.tex file which contains the majority of the structure behind the template

\begin{document}

%----------------------------------------------------------------------------------------
%	TITLE PAGE
%----------------------------------------------------------------------------------------


\maketitle


%----------------------------------------------------------------------------------------
%	TABLE OF CONTENTS
%----------------------------------------------------------------------------------------

%\chapterimage{11081241_828603353860317_4962848819839491087_n.jpg} % Table of contents heading image

\pagestyle{empty} % No headers

\tableofcontents % Print the table of contents itself

%\pagestyle{fancy} % Print headers again

\part{Fundamentals}
  \chapter{On Libertarian Communism and the effectiveness of Western Anarchists}
  
  \chapter{Introduction to terms}
    \epigraph{Thus it has come about that our theoretical and critical literature, instead of giving plain, straightforward arguments in which the author at least always knows what he is saying and the reader what he is reading, is crammed with jargon, ending at obscure crossroads where the author loses its readers. Sometimes these books are even worse: they are just hollow shells. The author himself no longer knows just what he is thinking and soothes himself with obscure ideas which would not satisfy him if expressed in plain speech.}{Carl von Clausewitz\\ \textit{On War}}
    We all know that various academic disciplines are dense with jargon, and this can make approaching these subjects to be quite intimidating. In this chapter, we will explain some basic terms from various fields - communist theory, military theory, marketing, etc. 
  \section{Social terms}
    \subsection{Privilege}
     Insert the material definition of privilege here, and expand on it.
    \subsection{Praxis}
    \paragraph{Praxis is the process by which a theory, lesson, idea, or skill is enacted, embodied, exercised, practiced, or realised.} A rather broad definition, "praxis" is both a noun and a verb, which covers the process of learning a concept, \textit{doing} the concept, and spreading it. For example, an apprenticeship is essentially praxis where the people involved are paid: Both the master and the apprentice; and the praxis is productive as well, in that the master-student pair work to make, repair, or service something.
    \subsection{Bourgeoisie}
      \paragraph{The bourgeoisie are the ruling upper class.} It includes the wealthy and the political elite. 
    \subsection{Proletariat}
      \paragraph{The proletariat are the working or unemployed lower class. Examples include janitors and factory workers.}
    \subsection{Petit-Bourgeoisie}
     The petit-bourgeoisie is the poor, relatively powerless property owner. They are still bourgeoisie in that they seek to become rich and powerful. Examples include small business owners and individual landlords.
    \subsection{Lumpenproletariat}
    \subsection{Class consciousness}
    \subsection{State}
     A state is the organised oppression of one class by another, with associated administrative apparatuses. For example, most modern-day states are bourgeois states; that is, they are the system of oppression of the working class by the bourgeoisie, along with the associated aministrative apparatuses, such as police and tax offices. States typically have a monopoly on violence.
  \section{Operational terms}
    \subsection{Schwerpunkt}
    \subsection{Auftragstaktik}

    \subsection{Friction}
      \epigraph{Everything in war is simple, but the simplest thing is difficult.}{Carl von Clausewitz\\ \textit{On War}}
      
  \chapter{What is oppression? On studying the effects of oppression versus studying on how to end it}
    \epigraph{The philosophers have only interpreted the world, in various ways. The point, however, is to change it.}{Karl Marx\\ \textit{Theses on Feuerbach}}
    \chapter{Levels of struggle and conflict}
    \section{Personal: the ideological}
      In the battle for
    \section{Cultural}
      Cultural struggle is the level of imperialism and colonialism. It is at this level where entire civilisations do battle, reguardless of whatever nation-state currently claims to represent them. 
    \section{Social}
      The social level of conflict is where society makes internal changes in its 
    \section{Political}
      \paragraph{The political level of conflict is the level of nation-states bickering and quarrelling, trade disputes, and so on.}
    \section{Strategic}
      \paragraph{The strategic level of conflict is the level of oil refineries, nuclear weapons, and factories. It can be thought of as the level which capitalism operates in order to support the political level.}
    \section{Theatre}
      \paragraph{The theatre level of conflict is the level of field armies, fronts, and supply dumps.}
    \section{Operational}
      \paragraph{The operational level of conflict is from the corps level down to the company level. An example is a battalion of infantry trying to secure a garrisoned village.}
    \section{Tactical}
      \paragraph{The tactical level is the platoon level down to the fire-team level. An example is a a platoon of troops securing a building.}
    \section{Personal: the combatant}
      \paragraph{The combatant level is the individual soldier or civilian. An example is an ordinary infantryman.}
      \epigraph{Of all the passions that inspire a man in a battle, none, we have to admit, is so powerful and so constant as the longing for honour and renown.}{Carl von Clausewitz\\ \textit{On War}}
  \chapter{Synthesism versus Platformism versus Insurrectionism}
    
    \chapter{The ego of a revolutionary}    
  	\epigraph{So it is said that if you know your enemies and know yourself, you can win a hundred battles without a single loss.
If you only know yourself, but not your opponent, you may win or may lose.
If you know neither yourself nor your enemy, you will always endanger yourself.}{Sun Tzu\\ \textit{The Art of War}}
    \epigraph{The general who advances without coveting fame and retreats without fearing disgrace, whose only thought is to protect his country and do good service for his sovereign, is the jewel of the kingdom.}{Sun Tzu}
    \epigraph{We must not become complacent over any success. We should check our complacency and constantly criticize our shortcomings, just as we should wash our faces or sweep the floor every day to remove the dirt and keep them clean.}{Mao Tse Tung}
 
  \chapter{Elements required for a successful revolution}
  \epigraph{Victorious warriors win first and then go to war, while defeated warriors go to war first and then seek to win.}{Sun Tzu}
	\section{Legitimacy}
    \epigraph{Socialism never took root in America because the poor see themselves not as an exploited proletariat but as temporarily embarrassed millionaires.}{John Steinbeck}
      \paragraph{Since the entire point of an anarchist revolution is to achieve full communism, naturally, the population must believe in the idea as a whole. Without it, the public would reject and shun those who try and liberate them. Therefore, the public as a whole must have a class consciousness. The difficulty of this will vary from place to place - for example, in the USA, a common slogan is "the poor see themselves not as an exploited proletariat, but as temporarily embarrassed millionaires". On the other hand, in rural Aboriginal Australian communities, they are used to being continually oppressed and harassed by the government, and are well aware of their situation. Thus, it is more likely that a successful anarchist revolution will take hold in rural Aboriginal communities, as opposed to major American cities.}
      \paragraph{Another factor which increases legitimacy is culture. A culture that is inherently classist, racist, sexist, homophobic, transphobic, ableist, etc, will see an anarchist revolution as an evil to stamp out. On the other hand, a culture that is inherently helping and fostering of camaraderie will see the revolution as a step forward.}
       
    \section{Organisation}
      \paragraph{Without organisation, anarchy quickly degenerates into chaos. Likewise, any attempt at installing anarchism will }
    \section{Post-revolution plan}
    \section{Logistics}
    \section{Matériel}
    \section{Training}
    \section{Opportunity}
  


\part{Revolutionary skills and knowledge}
    \chapter{On the need for constant study}
    \epigraph{Hungry man, reach for the book: it is a weapon.}{Bertolt Brecht}
    \epigraph{I'm very good at integral and differential calculus;\\
  I know the scientific names of beings animalculous:\\
  In short, in matters vegetable, animal, and mineral,\\
  I am the very model of a modern Major-General.}{Gilbert and Sullivan \\ \textit{The Pirates of Penzance}}
    \section{Operational research}
	  \paragraph{Operational research is the mathematical field of optimal decision making. It has a lot of overlap with game theory, control theory, and probability. Examples of problems addressed in operational research are optimal machine placement on factory floors, determining the best location to reinforce armour on military vehicles, and determining the optimal scheduling of logistic shipments.}
    \subsection{Management science}
    	\paragraph{Management science can be defined as operational research applied to personnel management settings.}
    \subsection{Game theory}
    	\paragraph{Game theory, and its offshoots such as drama theory and confrontation analysis, can be thought of as \textit{interactive} operational research, where there are multiple decision makers, each with their own (not necessarily conflicting) objectives. Examples include air traffic control routing, where the aim is the maximise the maximum distance between planes on the same air corridor; pursuit-and-evasion scenarios such as finding the best way for a missile to hit a certain target behaving in a certain way; and automated stock trading.}
    \section{Military theory}

    \section{Marketing}
      \subsection{Guerilla marketing}
    \section{Sociology}
    
      
  \chapter{Operational research, Game theory, and Decision making}
  \epigraph{The general who wins the battle makes many calculations in his temple before the battle is fought. The general who loses makes but few calculations beforehand.}{Sun Tzu}
    \section{OODA loop}
      \subsection{Observe}
      \subsection{Orient}
      \subsection{Decide}
      \subsection{Act}
    \section{PDCA loop}
      \subsection{Plan}
      \subsection{Do}
      \subsection{Check}
      \subsection{Act}

  \chapter{Organisation}
	\section{Auftragstaktik: Mission-type tactics, autonomy, its its appeal to Anarchists}
    \section{The fallacy of Insurrectionary and Individualist Anarchism}

  \chapter{Momentum, Morale, Manoeuvre and Mass: Basic principles of Strategic Action}
    \epigraph{It is better to go on striking in the same direction than to move one's forces this way and that.}{Carl von Clausewitz\\ \textit{On War}}
    

  \chapter{Analysis: Identifying the correct targets}
    \paragraph{Knowing what is even a problem is arguably the hardest part of acting. After all, is that not the reason why anarchism is so widely dismissed? In this chapter, we shall study some analysis techniques for determining what the problems are.}
    \section{STEEPLED analysis}
    	\subsection{Social factors}
        \subsection{Technological factors}
        \subsection{Environmental factors}
        \subsection{Economic factors}
        \subsection{Political factors}
        \subsection{Legal factors}
        \subsection{Ethical considerations}
        \subsection{Demographic factors}
        
        
   \chapter{Planning: Actions, campaigns, events, propaganda}
  	\section{SWOT analysis}
      \subsection{Internal factors}
    	\subsubsection{Strengths}
        \paragraph{Strengths are internal characteristics that give you an advantage over others. For example, a jet fighter could have an extremely high speed.}
        \subsubsection{Weaknesses}
        \paragraph{Weaknesses are internal characteristics that put you at a disadvantage relative to others. For example, the jet fighter could have a low manoeuvrability.}
      \subsection{External factors}
        \subsubsection{Opportunities}
        \paragraph{Opportunities are characteristics of your opponent or other circumstances that you can exploit to your advantage. For example, the jet fighter could be placed behind and above an enemy bomber.}
        \subsubsection{Threats}
        \paragraph{Threats are characteristics of your opponent or other circumstances that could prevent you from achieving your objective. For example, the enemy bomber might have its own fighter escort.}

        
  \chapter{Orders and coordination: Making sure everyone knows what to do}
    \section{Clarity vs confusion: Why plan?}
    \section{Considerations for good plans}
      From US Army FM-5.0
      \begin{description}[style=nextline]
        \item[Contain critical facts and assumptions]
          The commander and staff evaluate all facts and assumptions. They retain for future reassessment only those facts and assumptions that directly affect an operation’s success or failure. Assumptions are stated in OPLANs, but not in OPORDs.
        \item[Authoritative expression]
          The plan or order reflects the commander’s intention and will. Therefore, its language is direct. It unmistakably states what the commander wants subordinate commands to do.
        \item[Positive expression]
          Instructions in plans and orders are stated in the affirmative: for example, ``The trains will remain in the assembly area;'' instead of, ``The trains will not accompany the unit.'' As an exception, some constraints are stated in the negative: for example, ``Do not cross Phase Line Blue before H+2.''
        \item[Avoid qualified directives]
          Do not use meaningless expressions, such as, ``as soon as possible (ASAP).'' Indecisive, vague, and ambiguous language leads to uncertainty and lack of confidence. For example, do not use ``try to retain;'' instead, say ``retain until.'' Avoid using unnecessary modifiers and redundant expressions, such as ``violently attack'' or ``delay while maintaining enemy contact.'' Use ``attack'' or ``delay.'' Army doctrine already requires attacking violently and maintaining enemy contact during delays.
        \item[Balance]
          Balance centralized and decentralized control. The commander determines the appropriate balance for a given operation based on mission, enemy, terrain and weather, troops and support available, time available, and civil considerations (METT-TC). During the chaos of battle, it is essential to decentralize decision authority to the lowest practical level. Over centralization slows action and inhibits initiative. However, decentralized control can cause loss of precision. The commander constantly balances competing risks while recognizing that loss of precision is usually preferable to inaction.
        \item[Simplicity]
          Reduce all elements to their simplest form. Eliminate elements not essential to understanding. Simple plans are easier to understand. 
        \item[Brevity]
          Be clear and concise. Include only necessary details. Use short words, sentences, and paragraphs. Do not include material covered in SOPs (standing operating procedures). Refer to those SOPs instead. 
        \item[Clarity]
          Everyone using the plan or order must readily understand it. Do not use jargon. Eliminate every opportunity for misunderstanding the commander’s exact, intended meaning. Use acronyms unless clarity is hindered. Keep the plan or order simple. Use only doctrinal terms and graphics.  
        \item[Completeness]
          Provide all information required for executing the plan or order. Use doctrinal control measures that are understandable, and allow subordinates to exercise initiative. Provide adequate control means (headquarters and communications). Clearly establish command and support relationships. Fix responsibility for all tasks. 
        \item[Coordination]
          Provide for direct contact among subordinates. Fit together all battlefield operating systems (BOS) for synchronized, decisive action. Identify and provide for mutual support requirements while minimizing the chance of fratricide. 
        \item[Flexibility]
          Leave room for adjustments to counter the unexpected. The best plan provides for the most flexibility. 
        \item[Timeliness]
          Send plans and orders to subordinates in adequate time to allow them to plan and prepare their own actions. In the interest of timeliness, accept less than optimum products only when time is short.
      \end{description}
    \section{Five paragraph formats}
    \section{When a command structure is needed}
  
  \chapter{Action: Implementing the Schwerpunkt}
  	\section{Deception}
    \epigraph{All warfare is based on deception. Hence, when we are able to attack, we must seem unable; when using our forces, we must appear inactive; when we are near, we must make the enemy believe we are far away; when far away, we must make him believe we are near.}{Sun Tzu\\ \textit{The Art of War}}
    
    
  \chapter{Post-mortems and operational research}
  
  
  \chapter{Security}
    \epigraph{In making tactical dispositions, the highest pitch you can attain is to conceal them.}{Sun Tzu}
  
\part{Strategies and tactics}
  \chapter{Education and skill sharing}
  
  
  \chapter{Subverting the tools of the enemy to work against them}
  
  
  \chapter{Agitation, Propaganda, and the raising of class consciousness}
	\section{Symbols}
    \section{Memes: The poetry of the proletariat}
      \paragraph{One of the major features of social media sites are memes. What are memes, exactly? The idea comes from biology - it is "an idea, behaviour, or style that spreads from person to person within a culture". Examples include fashion trends, viral videos, pop-culture references, and chain emails. One particularly pervasive example, which we shall focus on, are captioned images. Image macros (i.e. a picture with text overlaid on it) and cartoons or comics are two examples of this type of meme. These can spread like wildfire on social media - indeed, one could say that they are a core part of what sets social media apart from traditional media.}
    
  \chapter{Liberals: Conversion of and working with}
  ``most important personal-political lesson of the last 2 years:
yeah, people got mixed-as-hell consciousness. it can be frustrating as fuck (and even dangerous on an individual level, which is why we organize). but if your politics are centered around a profound hatred of the masses then just admit you are interested in a subculture and not liberation please
and have the humility to admit that to yourself and realize that the social capital you are getting the consolatory benefits of is predicated on the isolation that is deadly to folks who can't afford to have a subculture in place of liberation.
i will take any salt-of-the-earth-type everyday working human with their messy interactions, lack of hipness to the social justice lingo, and basic willingness to learn ANYDAY over academic peddlers of hopelessness and ppl who gain idol status off of a post-modernist hate of the working masses.
You know who they are: they love to self-glorify in the model of individual education and call-out culture but are terrified of how actual radical education spaces attached to actual class struggle work will decentralize their individual selves.
get this: you aren't going to analyze oppression out of existence. say it again. you aren't going to analyze oppression out of existence. once more. analysis doesn't make a revolution. the masses do.
"Critiquing people where they are at is important, but we aren't gonna create the culture we want just by editing the culture capitalism creates." -Boots Riley
if we aren't talking about winning this thing, then we're done talking.'' - jamie saunders
  
  \chapter{Proving the illegitimacy of hierarchies, governments, and the ruling classes to ourselves and to the masses}
  
  
    \chapter{Callout culture and self-criticism}
	\section{The toxic effects of modern callout culture}
    	\epigraph{Another point that should be mentioned in connection with inner-Party criticism is that some comrades ignore the major issues and confine their attention to minor points when they make their criticism. They do not understand that the main task of criticism is to point out political and organizational mistakes. As to personal shortcomings, unless they are related to political and organizational mistakes, there is no need to be overcritical or the comrades concerned will be at a loss as to what to do. Moreover, once such criticism develops, there is the great danger that within the Party attention will be concentrated exclusively on minor faults, and everyone will become timid and overcautious and forget the Party's political tasks.}{Mao Tse Tung}


  \chapter{The Struggle for the Lumpenproletariat to be revolutionaries}
  Activists need support. You know who don't get much support in society? The lumpen. That's why it's harder for them to spend energy on being activists.
  
  
  \chapter{Demands versus objectives}
    Demands give the impression that the bourgeoisie have something that we want, and we are trying to get it from them; whereas objectives implies we are taking what we want by ourselves.

  \chapter{Violence, resistance, and oppression}

  \chapter{Public relations and the image of Anarchism}
    \section{Graffiti}
    \section{Riots}


  \chapter{Violence, terrorism, and coercion: Immoral and counter-productive}


  \chapter{Solidarity, forgiveness, sectarianism, and working together}



\part{Other topics}




\subsection{References}\index{References}

Since I found so much good information about pretty much everything I wanted to know about, I will just create a remark and let you know where you can find more specific information about, just like below.

%\begin{remark}
%Only the rarest, dankest, and most illegal of memes will subvert the public consciousness
%\end{remark}

\part{Appendices}
\input{references/secretwork.tex}
\end{document}