\chapter{Introduction to terms}
    \epigraph{Thus it has come about that our theoretical and critical literature, instead of giving plain, straightforward arguments in which the author at least always knows what he is saying and the reader what he is reading, is crammed with jargon, ending at obscure crossroads where the author loses its readers. Sometimes these books are even worse: they are just hollow shells. The author himself no longer knows just what he is thinking and soothes himself with obscure ideas which would not satisfy him if expressed in plain speech.}{Carl von Clausewitz\\ \textit{On War}}
    We all know that various academic disciplines are dense with jargon, and this can make approaching these subjects to be quite intimidating. In this chapter, we will explain some basic terms from various fields - communist theory, military theory, marketing, etc. 
  \section{Social terms}
    \subsection{Privilege}
     Insert the material definition of privilege here, and expand on it.
    \subsection{Praxis}
    \paragraph{Praxis is the process by which a theory, lesson, idea, or skill is enacted, embodied, exercised, practiced, or realised.} A rather broad definition, "praxis" is both a noun and a verb, which covers the process of learning a concept, \textit{doing} the concept, and spreading it. For example, an apprenticeship is essentially praxis where the people involved are paid: Both the master and the apprentice; and the praxis is productive as well, in that the master-student pair work to make, repair, or service something.
    \subsection{Bourgeoisie}
      \paragraph{The bourgeoisie are the ruling upper class.} It includes the wealthy and the political elite. 
    \subsection{Proletariat}
      \paragraph{The proletariat are the working or unemployed lower class. Examples include janitors and factory workers.}
    \subsection{Petit-Bourgeoisie}
     The petit-bourgeoisie is the poor, relatively powerless property owner. They are still bourgeoisie in that they seek to become rich and powerful. Examples include small business owners and individual landlords.
    \subsection{Lumpenproletariat}
    \subsection{Class consciousness}
    \subsection{State}
     A state is the organised oppression of one class by another, with associated administrative apparatuses. For example, most modern-day states are bourgeois states; that is, they are the system of oppression of the working class by the bourgeoisie, along with the associated aministrative apparatuses, such as police and tax offices. States typically have a monopoly on violence.
  \section{Operational terms}
    \subsection{Schwerpunkt}
    \subsection{Auftragstaktik}

    \subsection{Friction}
      \epigraph{Everything in war is simple, but the simplest thing is difficult.}{Carl von Clausewitz\\ \textit{On War}}